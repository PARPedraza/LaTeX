%------------------------------------------------------------
%SOLUCIÓN — Ejercicio 04: Documento mínimo
%------------------------------------------------------------

%Archivo: solucion_ejercicio04.tex

%------------------------------------------------------------
%Código de ejemplo:
%------------------------------------------------------------

\documentclass{article}
\usepackage[utf8]{inputenc}
\usepackage[T1]{fontenc}
\usepackage[spanish]{babel}

\title{Documento mínimo en LaTeX}
\author{Tu Nombre}
\date{\today}

\begin{document}
\maketitle

\section{Introducción}
Este es un documento mínimo de ejemplo.
Muestra los elementos esenciales para compilar un
archivo en LaTeX: la clase, el preámbulo y el entorno
principal del documento.

\end{document}

%------------------------------------------------------------
%Instrucciones:
%------------------------------------------------------------
%1. Copia el código en un archivo llamado
%   estructura_minima.tex
%2. Compílalo con pdfLaTeX o desde Overleaf.
%3. Observa cómo se generan automáticamente
%   el título, autor, fecha y la primera sección.
%------------------------------------------------------------


